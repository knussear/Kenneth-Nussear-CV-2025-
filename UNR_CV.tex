%%%%%%%%%%%%%%%%%%%%%%%%%%%%%%%%%%%%%%%%%
% Medium Length Professional CV
% LaTeX Template
% Version 2.0 (8/5/13)
%
% This template has been downloaded from:
% http://www.LaTeXTemplates.com
%
% Original author:
% Trey Hunner (http://www.treyhunner.com/)
%
% Important note:
% This template requires the resume.cls file to be in the same directory as the
% .tex file. The resume.cls file provides the resume style used for structuring the
% document.
%
%%%%%%%%%%%%%%%%%%%%%%%%%%%%%%%%%%%%%%%%%

%----------------------------------------------------------------------------------------
%	PACKAGES AND OTHER DOCUMENT CONFIGURATIONS
%----------------------------------------------------------------------------------------
\title{Kenneth Nussear CV}
%\documentclass[a4paper,skipsamekey,11pt,british]{curve}
\documentclass{resume} % Use the custom resume.cls style
\usepackage[dvipsnames]{xcolor}
\usepackage[left=0.75in,top=1in,right=0.75in,bottom=1in]{geometry} % Document margins
\usepackage{enumitem}
\usepackage{url}
\makeatletter
\g@addto@macro{\UrlBreaks}{\UrlOrds}
\makeatother

\newcommand{\tab}[1]{\hspace{.2667\textwidth}\rlap{#1}}
\newcommand{\itab}[1]{\hspace{0em}\rlap{#1}}
\usepackage{fancyhdr}
\pagestyle{fancy}
\makeatletter
\makeatother
%-----------------------

%----------------- another bibtex

\usepackage[sorting=ydnt,style=authoryear,url = false, defernumbers=true, isbn=true, maxbibnames=12, language=american, backend=biber, dashed=false]{biblatex}

\addbibresource[label=bibdb]{ZoteroJA.bib}

%-----------------


\name{Kenneth E. Nussear, PhD} % Your name
\address{Department of Geography \\ University of Nevada \\ Reno, NV 89557} % Your address
\address{web: https://www.researchgate.net/profile/Kenneth-Nussear}
\address{(775)784-6612 \\ knussear@unr.edu} % Your phone number and email


\renewenvironment{rSection}[1]{
\sectionskip
\textcolor{MidnightBlue}{\MakeUppercase{#1}}
\sectionlineskip
\hrule
\begin{list}{}{
\setlength{\leftmargin}{1.5em}
}
\item[]
}{
\end{list}
}


\begin{document}
%----------------------------------------------------------------------------------------
%	Professional Interests
%----------------------------------------------------------------------------------------
\begin{rSection}{Professional Interests}
My research program seeks to understand both human-caused and natural influences on the geographic distribution of species and their habitat, and the connectivity and demographics of populations impacted by habitat change resulting in fragmentation and degradation. My research integrates principles of Conservation Biology, Wildlife Biology and Management with Spatial Ecology so that we can gain a richer understanding of what challenges species face, and where on the landscape different influences and management decisions are likely to be important and effective. 
\end{rSection}

%----------------------------------------------------------------------------------------
%	EDUCATION SECTION
%----------------------------------------------------------------------------------------

\begin{rSection}{Education}

{\bf University of Nevada, Reno} \hfill
\\ PhD Ecology, Evolution, and Conservation Biology \hfill {\em 2004} 
\begin{itemize}
\item[$\cdot$] \textbf{} Mechanistic investigation of the distributional limits of the desert tortoise \textit{Gopherus agassizii}. \textbf{\textit{Advisor:}} \textit{C. Richard Tracy} \hfill
\end{itemize}
{\bf Colorado State University} \hfill
\\ BS Zoology \textit{Summa Cum Laude}\hfill {\em 1995} 
\end{rSection}

%----------------------------------------------------------------------------------------
%	WORK EXPERIENCE SECTION
%----------------------------------------------------------------------------------------
\begin{rSection}{Professional Experience}

  \begin{rSubsection}{Academic Appointments}{ }{ }{ }
  \vspace{-0.8em}
    \textbf{University of Nevada, Reno}\\
    \textit{Associate Professor} – Department of Geography \hfill {\em 2022 to Present}\\
    \textit{Assistant Professor} – Department of Geography \hfill {\em 2017 to 2022}\\
    \textit{Research Professor} – Department of Geography \hfill {\em 2016 to 2017}\\
    \textit{Adjunct Assistant Professor} – Department of Geography \hfill {\em 2009 to 2017}\\
    \textit{Adjunct Assistant Professor} – Department of Biology \hfill {\em 2009 to 2017}\\
    \textit{Postdoctoral Fellow} – Department of Biology \hfill {\em 2004}\\
    \textit{Graduate Research Assistant} – Department of Biology \hfill {\em 1995 to 2004}\\
    \textit{Webmaster and Online Database Administration} – Biological Resources Research Center \hfill {\em 1995 to 2006}\\
    \textbf{University of Nevada, Las Vegas}\\
    \textit{Adjunct Assistant Professor} – Department of Environmental Studies \hfill {\em 2009 to 2014}
  \end{rSubsection}

  \begin{rSubsection}{Private Industry}{ }{ }{ }
  \vspace{-0.8em}
    \textit{Owner - Spatial Ecologist} – Southwest Ecology LLC \hfill {\em 2015 to Present}
  \end{rSubsection}

  \begin{rSubsection}{US Government}{ }{ }{ }
  \vspace{-0.8em}
    \textit{Research Wildlife Biologist} – US Geological Survey, Western Ecological Research Center \hfill {\em 2004 to 2014}
  \end{rSubsection}

\end{rSection}
%------------------------------------------------
% Presentations
%------------------------------------------------
\begin{rSection}{Professional Presentations}
\begin{refsection}[bibdb]
%\rhead{Nussear - Curriculum Vitae \hfill Presentations Cont.}

\nocite{*}
%\addbibresource{Presentations.bib}
\defbibnote{prnote}{(* Student Speaker, ** Presenter)}
\printbibliography[heading={subbibliography},title={Presentations},type=inproceedings, prenote=prnote, keyword={presentation}]
\end{refsection}
\begin{refsection}[bibdb]
\nocite{*}
%\pagebreak
%\addbibresource{InvitedPresentations.bib}
\defbibnote{prnote}{(* Student Speaker)}
\printbibliography[heading={subbibliography},title={Invited Presentations},type=inproceedings, prenote=prnote, keyword={invited}]
\end{refsection}
%\rhead{Nussear - Curriculum Vitae \hfill Professional Presentations Cont.}
\begin{rSubsection}{Media}{}{}{}
\item Appearances and Interviews
\begin{itemize}
\item 2019 \textit{“Remembering Normandy, a commemorative jump” - Nevada Today}\\
Interview and appearance \\ 
\url{https://www.unr.edu/nevada-today/news/2019/remembering-normandy}
\item 2014 \textit{“Tortoise in Peril” - Topic Productions, Inc.}\\
Interview and appearance \\ 
\url{https://vimeo.com/116259592}
\item 2013 \textit{KQED Quest Sustainability Science – Tv Series Interview} \\ \url{http://ww2.kqed.org/quest/2013/11/05/largest-solar-plant-in-the-world-goes-through-last-test-before-opening/}

\item 2013 \textit{One:Eight Productions - “60 Million Years”} Interview and appearance \\
\url{https://vimeo.com/100119638}
\item 2009 to 2010 \textit{USGS Documentary “The Heat Is On: Desert Tortoises and Survival”} \\Participated in production and interviewed extensively in \url{http://pubs.usgs.gov/gip/98}. Provided follow-on presentations for the 20th Anniversary of USGS Evening Public Lecture Series, Menlo Park, CA, 5-27-2010. [2 Joint presentations]
\item 2008 \textit{National Geographic Kids magazine} \\
Interview for Wildlife Watch feature about research that compared the ability of trained working dogs with that of humans in finding desert tortoises.
\item 2008 \textit{Press-Enterprise} Interview for a story about the relocation of desert tortoises that will lose their habitat due to the planned expansion at Fort Irwin \\ \url{http://www.pe.com/localnews/environment/stories/PE_News_Local_D_tortoise28.3bcbad4.html}
\item 2008\textit{ Sierra Magazine - The Tortoise and the Hare} \\
Interviewed in article about global warming and wildlife \\ \url{http://vault.sierraclub.org/sierra/200805/tortoise.asp}
\item 2006 \textit{The Desert Speaks - "On the trail of a living fossil"} \\ Interview and appearance \\
\url{http://www.klru.org/schedule/episode/136966/}
\end{itemize}

\end{rSubsection}

\end{rSection}
%------------------------------------------------
% Grants
%------------------------------------------------
\begin{rSection}{Grants}{}
\begin{itemize}
\item Clark County Desert Conservation Program (\$80K) \hfill 2025\\
Covered Species Model Updates -- updates of habitat models for 13 species of conservation concern. Project includes training of graduate and undergraduate students
\textit{Pi - K. Nussear}
\item Mohave Ground Squirrel Conservation Council (\$225K) \hfill 2024\\
Mohave Ground Squirrel Long-Term Monitoring -- Using UAS to survey for MGS, and characterize habitat and vegetation. Project includes training of graduate students
\textit{Pi - K. Nussear}
\item Clark County Desert Conservation Program (\$79K) \hfill 2023\\
Covered Species Model Updates -- updates of habitat models for 17 species of conservation concern. Project includes training of graduate and undergraduate students
\textit{Pi - K. Nussear}
\item Heritage Environmental (\$45K) \hfill 2022-2023\\
Desert Tortoise Population Connectivity in the Vicinity of the Proposed Bonanza Solar Project - evaluating potential impacts of solar development to tortoise populations near Indian Springs, Nevada.
\textit{Pi - K. Nussear}

\item Conservation Science Partners (CSP) (\$93.5K) \hfill 2022 to 2024 \\
A spatially explicit population model for desert tortoises -- Assessment and analysis of historical desert tortoise data toward a range wide demographic model.
\textit{Pi - K. Nussear}

\item Bureau of Land Management (BLM) (\$146K) \hfill 2021 to 2023 \\
Demographics and habitat change in the Desert Tortoise -- Demographic capture recapture analyses in the Mojave desert of California.
\textit{Pi - K. Nussear}

\item National Fish and Wildlife Foundation (NFWF) (\$2M) \hfill 2020 to 2030 \\
Conservation Corridors for Desert Tortoises -- Landscape scale monitoring of connectivity of desert tortoise populations through natural and man made corridors. Comparing demographic and genetic connectivity through a diverse array of habitats affected by utility scale solar development. \textit{Pi - K. Nussear}

\item Clark County Desert Conservation Program (\$248K
) \hfill 2020 to 2021 \\
Desert Tortoise Connectivity Solutions Modeling
 -- Modeling current and future desert tortoise habitat and responses to land use and climate change to identify ways to restore and/or maintain connectivity within Clark County into the future. \textit{Pi - K. Nussear}

\item Clark County Desert Conservation Program (\$37K
) \hfill 2020 to 2022 \\
Gila Monster Habitat Modeling
 -- Modeling habitat for the Gila monster throughout Clark County, NV in current, and incorporating dispersal data to predict potential changes in habitat due to climate change. \textit{Pi - K. Nussear}

\item Bureau of Land Management (\$188K
) \hfill 2020 to 2022 \\
Application of UAS to supplement monitoring of threatened and endangered plant species in desert environments
-- Using UAS to locate and quantify populations of Peirson's milk-vetch on the Imperial Sand Dunes in California. \textit{Pis - K. Nussear and S. Bassett}

\item Bureau of Land Management (\$267K 
) \hfill 2020 to 2021 \\
Desert Sunlight Tortoise Genetic Connectivity Study
 -- Evaluating genetic connectivity of desert tortoise populations in the landscape in an around Desert Center, CA and the Desert Sunlight Solar Energy Facility. \textit{Pi - K. Nussear}

\item US Geological Survey (\$20K)
 \hfill 2019 to 2022 \\
Spatial analysis support for Ft Irwin Translocation planning
-- Evaluation and spatial analysis of habitat for the potential translocation of desert tortoises. \textit{Pi - K. Nussear}

\item Clark County Desert Conservation Program (\$126K
) \hfill 2019 to 2020 \\
Species Distribution Modeling, Phase II
 -- Modeling species habitat and spatial analysis of effects of proposed development areas and designated reserve areas for 17 covered species under the Clark County Desert Habitat Conservation Plan. \textit{Pi - K. Nussear}
 \item Clark County Desert Conservation Program (\$10.6K
) \hfill 2019\\
Golden Eagle Habitat Model 
 -- Modeling species habitat and spatial analysis of effects of proposed development areas on Golden Eagle nesting and overall habitat for the Clark County Desert Habitat Conservation Plan. \textit{Pi - K. Nussear}
 
\item National Fish and Wildlife Foundation (NFWF) (\$91K) \hfill 2018 to 2019 \\
Spatial Analysis of Raven Monitoring and Management Data for Desert Tortoise Critical Habitat Units in the Mojave Desert of California. 
-- Analysis of temporal data on nesting characteristics of the Common Raven in the Mojave desert of California, and their potential influence on populations of Desert Tortoises. \textit{Pi - K. Nussear}

\item University of Nevada, Reno (\$5K) \hfill 2018 \\
Instructional Improvement Grant- Geospatial Database servers and software for advanced instruction. \textit{Pis - K. Nussear and S. Kelley}

\item Strategic Environmental Research and Development Program (SERDP) (\$1.69M) \hfill 2018 \\
The impacts of land use and climate change on Mojave Desert tortoise gene flow and corridor functionality.
-- The objective of this research is to determine how land use and climate change will impact Mojave Desert tortoise gene flow and corridor functionality within the context of multi-species interactions and landscape connectivity of military installation and other federal lands. \textit{Co-Applicants: Jill Heaton, K. Nussear, Doug Boyle, Scott Bassett, Marjorie Matocq - UNR, and Todd Esque, Amy Vandergast - USGS}

\item Clark County Desert Conservation Program (\$95K
) \hfill 2017 to 2019 \\
Desert tortoise connectivity modeling
 -- Development of genetic connectivity models to identify the potential influence of habitat change and barriers to movement on long term genetics of desert tortoises. \textit{Pis Jill Heaton, K. Nussear, and Kirsten Dutcher - (PhD Student)}.

\item University of Nevada, Reno (\$5K) \hfill 2017 \\
Instructional Improvement Grant- Unmanned Aerial systems for Geography Students. \textit{Pis - K. Nussear and S. Bassett}

\item Clark County Desert Conservation Program (\$460K) \hfill 2016 to 2017 \\
Covered species support -- Development of Species Distribution Models for updated to the Clark County Multiple Species Habitat Conservation Plan. Spatial analysis of effects of proposed development areas and designated reserve areas to 56 species of plants and animals. \textit{Pi - K. Nussear}
\item National Fish and Wildlife Foundation (\$1.5M) \hfill 2015 to 2020 \\ 
Conservation Corridors for Desert Tortoises -- Landscape scale monitoring of connectivity of desert tortoise populations through natural and man made corridors. Comparing demographic and genetic connectivity through a diverse array of habitats affected by utility scale solar development. \textit{Pi - K. Nussear}
\item Universität Konstanz -- Max Planck Institute (\$30K) \hfill 2015 to 2016 \\ 
Programming ecological models for 220 African Bat species in R to create a framework to calculate distance matrices at a large scale, that can be input for metapopulation capacity calculations in support of “Prioritizing habitat conservation and restoration for bats in continental Africa based on high-resolution distribution models”. Scientific programming of large datasets using multicore and parallel programming R and SGE clusters to calculate spatial distances and metapopulation extinction estimates. \textit{Pi - K. Nussear}
\item Bureau of Land Management (\$400K) \hfill 2013 to 2014  \\ 
Conservation Corridors for Desert Tortoises – Landscape scale monitoring of connectivity of desert tortoise populations through natural and man made corridors. Comparing demographic and genetic connectivity through a diverse array of habitats affected by utility scale solar development. \textit{Pi - K. Nussear}
\item US Fish and Wildlife Service and Coyote Springs LLC (\$300K) \hfill 2013 to 2014 \\ Influences of habitat quality on gene expression in desert tortoises translocated to burned habitat. Comparing habitat influences on the physiology of tortoises translocated to burned and unburned habitat. \textit{Pis - K. Nussear and T. Esque}
\item National Science Foundation (\$1.25M) \hfill 2012 \\ 
Epidemiology of Mycoplasmal Upper Respiratory Tract Disease in Desert Tortoises – Evaluating the potential for established social networks to be disrupted by using network models to describe the contacts of individuals, and the rate at which contacts become transmission events. \textit{Co-Applicant with Peter Hudson, Penn State University, and K. Nussear, USGS}
\item Bureau of Land Management (\$50K) \hfill 2012 \\ 
Epidemiology of Mycoplasmal Upper Respiratory Tract Disease in Desert Tortoises – Evaluating the potential for established social networks to be disrupted by using network models to describe the contacts of individuals, and the rate at which contacts become transmission events. \textit{Pis - K. Nussear and T. Esque}
\item Department of the Army (\$840K)  \hfill 2012 \\ Habitat quality assessments of and gene expression of tortoises at Ft. Irwin NTC. \textit{Pis - K. Nussear, T. C. Esque and P. A. Medica}
\item Bureau of Land Management and First Solar (\$300K) \hfill 2011 \\ 
Conservation Corridors for Desert Tortoises – studying population connectivity through mountain passes \textit{Pi - K. Nussear}
\item Department of the Army (\$1.1M) \hfill 2011 \\ 
Habitat quality assessments of and gene expression of tortoises at Ft. Irwin NTC and Desert tortoise translocation. \textit{Pis - K. Nussear, T. C. Esque, and P. A. Medica}
\item Bureau of Land Management (\$200K) \hfill 2010 \\ 
The implications of climate change scenarios on the potential habitat of the desert tortoise in the Mojave and Sonoran Deserts of the Southwestern United States. \textit{Pis - K. Nussear and T. C. Esque}.
\item Bureau of Land Management (\$140K) \hfill 2010 \\ 
Nutritional Ecology of Desert Tortoises: a study of the effects dust suppressants on forage and the nutrition and physiology of desert tortoises: \textit{Pis - K. Nussear, T. C. Esque, and P. A. Medica}
\item Department of the Army (\$1.1M) \hfill 2010 \\ 
Stress physiology of resident and translocated tortoises being translocated at the Ft. Irwin NTC. \textit{Pis - K. Nussear, T. C. Esque, and P. A. Medica}
\item Bureau of Land Management (\$75K) \hfill 2009 \\ 
Nutritional Ecology of Desert Tortoises: a study of the effects dust suppressants on forage and the nutrition and physiology of desert tortoises: \textit{Pis - K. Nussear, T. C. Esque, and P. A. Medica}
\item U.S. Geological Survey (\$800K) \hfill 2009 \\ 
Impacts of solar energy development to wildlife movement and population persistence in the desert southwest. \textit{PIs - K. Nussear, A. Vandergast, T. C. Esque, and R. Fisher}
\item Bureau of Land Management (\$180K) and CSI conservation fund (\$127,000) \hfill 2008 \\ 
Responses of desert tortoises to post-wildfire habitat. \textit{Pis - K. Nussear, T. C. Esque, and P. A. Medica in collaboration with C. R. Tracy and S. Snyder - University of Nevada, Reno}
\item CSI conservation fund (\$65K) \hfill 2008 \\ 
Nutritional Ecology of Desert Tortoises: a study of the effects of exotic forage on the nutrition and physiology of desert tortoises. \textit{Pis - K. Nussear, T. C. Esque, and P. A. Medica in collaboration with C. R. Tracy - University of Nevada, Reno}
\item Department of the Army (\$913K) \hfill 2008 \\ 
Stress physiology of resident and translocated tortoises being translocated at the Ft. Irwin NTC. \textit{Pis - K. Nussear, T. C. Esque and P. A. Medica in collaboration with C. R. Tracy, K. Hunter, A. Barber, and S. DuPre - University of Nevada, Reno}
\item Bureau of Land Management (\$750K) \hfill 2008 \\ 
Multispecies habitat modeling for Gold Butte management area. \textit{Pis - K. Nussear, T. C. Esque, L. A. DeFalco, and K. Longshore, in collaboration with J. S. Heaton - University of Nevada, Reno}
\item Desert Research Institute (\$75K) \hfill 2008 \\ 
Validation and Development of a Certification Program Using K9s to Survey Desert Tortoises. \textit{Pis - K. Nussear, and T. C. Esque}
\item Department of the Army (\$742K) \hfill 2007 \\ 
Stress physiology of resident and translocated tortoises prior to translocation at the Ft. Irwin NTC. \textit{Pis - K. Nussear, T. C. Esque and P. A. Medica in collaboration with C. R. Tracy, K. Hunter, A. Barber, and S. DuPre - University of Nevada, Reno}
\item Desert Research Institute, (\$40K) \hfill 2006 \\ 
Validation and development of a certification program for using K9s to survey desert tortoises, \textit{Pis - K. Nussear, and T. C. Esque in collaboration with lead investigators M. Cablk, DRI, and J. S. Heaton, UNR}
\item Department of the Army (\$475K) \hfill 2006 \\ 
Ecology of resident tortoises prior to translocation at the Ft. Irwin NTC. \textit{Pis - K. Nussear, T. C. Esque and P. A. Medica}
\item Bureau of Land Management (\$99K)  \hfill 2006 \\
Reptiles on the Urban Edge: a study of the influences of human impacts on reptile communities in the Sloan Canyon National Conservation and Wilderness Areas, NV. \textit{Pi - K. Nussear}
\item US Fish and Wildlife Service Desert Tortoise Recovery Office, (\$45K)  \hfill 2005 \\
Coordination and technical assistance with range-wide monitoring for desert tortoises, and monitoring of focal populations for behavioral quantification. \textit{Pi - K. Nussear}
\item Department of the Army (\$205K)  \hfill 2005 \\
Ecology of resident tortoises prior to translocation at the Ft. Irwin NTC. \textit{Pis - K. Nussear, T. C. Esque and P. A. Medica}
\item Department of the Army, (\$402K)  \hfill 2005 \\
Comparisons of human and canine assisted methods for clearances of desert tortoises. \textit{Pis - K. Nussear, T. C. Esque and P. A. Medica; in collaboration with J. S. Heaton, UNR and M. Cablk, DRI}
\item University of Nevada Reno, (\$50K)  \hfill 2005 \\
Multiple species habitat conservation planning database administration and development. \textit{Pi - K. Nussear}
\item Department of the Army, (\$75K)  \hfill 2005 \\
Translocation plan for Ft. Irwin NTC. \textit{Pi - K. Nussear, T. C. Esque and P. A. Medica}
\item Pre-Doctoral Fellowship, University of Redlands, Redlands Ca. (\$50K) \hfill 2003 \\ Modeling Tortoise activity (g0) as a function of the environment, with applications toward distance sampling. \textit{Pi - K. Nussear}
\item Sigma Xi: Grant-in Aid of Research (\$400) 
\hfill 1995 \\ 
Locomotory performance in anurans: the challenges of metamorphosis.\textit{Pi - K. Nussear}
\end{itemize}
\end{rSection}
%----------------------------------------------------------------------------------------
%	TECHNICAL STRENGTHS SECTION
%----------------------------------------------------------------------------------------
\begin{rSection}{Post Graduate Training}
\begin{itemize}
\item Mentoring Mentors Workshop, UNR \hfill 2024
\item Diversity, Equity, and Inclusion in STEM Workshop Series \hfill 2022
\item Certificate in Effective Instruction \hfill 2019 \\
  Association of College and University Educators (ACUE)
\item FAA Certified Remote Pilot for UAS \hfill 2016
\item FAA Certified Sport Pilot \hfill 2011
\item USFWS Desert Tortoise Health Assessment Training, Las Vegas NV, March 8-10. \hfill 2011 
\item Bayesian Statistics Short Course, Sacramento, California, Feb 10-12. \hfill 2011
\item Structural Equation Modeling, Sacramento, California, April 21-22. \hfill 2008
\item USFWS Jugular and Subcarapacial Venipuncture, Las Vegas, NV. \hfill 2007
\item Applied Spatial Statistics in Ecology – Sacramento, California, January 24-26. \hfill 2007
\item Information-Theoretic Methods -- Statistical Inference using AIC \hfill 2006 \\
 and Multimodel Selection. \\ Sacramento, California, March 15-16. 
\item USGS Media Training, Sacramento, CA. \hfill 2004
\end{itemize}
\end{rSection}
\pagebreak
\begin{rSection}{Computing and Analytical Skills}
%\rhead{Nussear - Curriculum Vitae \hfill Data Analytics Skills Cont.}
%\begin{tabular}{ @{} >{\bfseries}l @{\hspace{6ex}} l }
\begin{tabular}{p{0.35\linewidth}p{0.6\linewidth}}
GIS & R GIS, QGIS, Grass GIS, GDAL, PostgreSQL, ArcGis \\
\tabularnewline
Programming Languages & R, Java, C++, Basic, Fortran, Pascal, Perl, Bash, VBA, html, Javascript, php and css\\
\tabularnewline
Parallel Computing & Sun Grid Engine (SGI) and Beowulf Clusters, MPI, SNOW, MPich, Parallel and Multi--core analysis in R\\
\tabularnewline
Other Software \& Tools &  LaTeX, Bibtex, Bibdesk, GIMP, Imagej, JMP, Minitab, Excel, ODBC, Filemaker, SQL, SQLite, DBF, Pendragon Forms, MS Access\\
\end{tabular}
\end{rSection}
%------------------------------------------------
% Students
%------------------------------------------------
\begin{rSection}{Postdoctoral/Research Associates Advised}
\begin{itemize}

\item Kirsten Dutcher, PhD. Post-Doctoral Associate 2020 - 2024. Department of Geography, University of Nevada, Reno.
 Quantifying genetic connectivity in and around the Eagle Mountain solar facility in Desert Center CA., and modeling the impacts of barriers on genetic connectivity in Clark County NV
 
 \item Eric Simandle, PhD. Research Associate 2019 - 2020. Department of Geography, University of Nevada, Reno.
 Species distribution modeling for 18 species for the Clark County NV, Desert Conservation Program.
 \end{itemize}
 \end{rSection}
\begin{rSection}{Graduate Students Advised}

\begin{rSubsection}{Committee Chair}{}{}{}
\begin{itemize}

\item[] Hannah Potts, MS Student – 2023 - Present. Department of Geography, University of Nevada, Reno. Using UAS to detect audio calls of the endangered Mohave Ground Squirrel.
\item[] Faith Machuca, MS Student – 2023 - Present. Department of Geography, University of Nevada, Reno. Disturbance and Change Detection in Desert Desert Landscapes of the Mojave.
\item[] Griffin Pugh, MS Student – 2024 - Present. Program in Ecology Evolution and Conservation Biology, University of Nevada, Reno. Desert tortoise demographics and vital rates. Awarded Linda Allison Grant - Desert Tortoise Council 2025
\item[] Jacqueline Dougherty, MS Student – 2021 - Present. Department of Geography, University of Nevada, Reno. Desert tortoise demographic modeling. Awarded Linda Allison Grant - Desert Tortoise Council 2023
\item[] Harrison York, PhD Student – 2024 - Present. Program in Ecology Evolution and Conservation Biology, University of Nevada, Reno. Influences of anthropogenic development on the demographics of desert tortoise and population connectivity.
\item[] Rose Ononeme, PhD Student – 2024 - Present. Department of Geography, University of Nevada, Reno. Effects of human disturbance on movements of wild desert tortoises

\\
\textit{Graduated}

\item[] Courtney Butt, MS 2024. Program in Ecology Evolution and Conservation Biology, University of Nevada, Reno. Decision support for mapping land for desert conservation priorities.

\item[] Anjana Parandhaman, PhD 2023. Program in Ecology Evolution and Conservation Biology, University of Nevada, Reno. The Impacts of Climate and Land Use Change on Mojave Desert Tortoise (Gopherus agassizii) Habitat Suitability and Landscape Genetic Connectivity. Awarded Dave Morafka Research Award - Desert Tortoise Council - 2021

\item[] Justice Best, MS 2023. Department of Geography, University of Nevada, Reno. Characterizing habitat and densities of the Mojave desert tortoise (Gopherus agassizii) at multiple spatial scales. Best Student Presentation Desert Tortoise Council 2023, Second Place - UNR Three Minute Thesis 2023

\item[] Steve Hromada, PhD 2022. Program in Ecology Evolution and Conservation Biology, University of Nevada, Reno. The genes must flow: using movement ecology to understand connectivity of Mojave desert tortoise (Gopherus agassizii) populations in altered landscapes. Awarded Dave Morafka Research Award - Desert Tortoise Council - 2020

\item[] Corey Mitchell, MS – 2020. Department of Geography, University of Nevada, Reno. Righting the wrongs using data from space: Desert Tortoises, demographics, and violated assumptions. Best Student Presentation ASIH 2019

\item[] Ally Xiong, MS 2020. Department of Geography, University of Nevada, Reno. Spatial analysis of Common Raven monitoring and management data for Desert Tortoise critical habitat units in California.
\end{itemize}
\end{rSubsection}

\begin{rSubsection}{Co Advisor}{}{}{}
\begin{itemize}
\item[] Cas Carroll, PhD Candidate – 2020 – Present. Program in Ecology Evolution and Conservation Biology, University of Nevada, Reno.  Habitat and genetics of the Carson wandering Skipper. \textit{Co-Advisor: Matt Forister}

\item[] Jeff Best, PhD Candidate – 2018 – Present. Department of Geography, University of Nevada, Reno.  Using UAS in applications of ecology, landscape conditions, and search and rescue. \textit{Co-Advisor: Scott Bassett}


\textit{Graduated}

\item[] Kirsten Dutcher, PhD 2020. Department of Geography, University of Nevada, Reno. Connecting the Plots: Anthropogenic Disturbance and Mojave Desert \textit{(Gopherus agassizii)} Genetic Connectivity. \textit{Co-Advisor: Jill Heaton}

\item[] Peter Van Linn. MS 2011. Environmental Science. University of Nevada, Las Vegas. Estimating wildfire potential on a Mojave Desert landscape using remote sensing and field sampling. \textit{Co-Advisor: Scott Abella}

\item[] Richard D. Inman, MS 2008. Department of Biology, University of Nevada, Reno. How elusive behavior and climate influence the precision of density estimates of desert tortoise populations. \textit{Co-Advisor: C. Richard Tracy}

\end{itemize}
\end{rSubsection}
\begin{rSubsection}{Committee member}{}{}{}
\begin{itemize}
\item[] Tyler S Pelletier PhD Student - 2025 - Present, Ecology, Evolution and Conservation Biology, University of Nevada, Reno. Health and seedling distributions of Whitebark Pine (\textit{Pinus albicaulis}) in relation to wildfires in California and Great Basin \textit{Advisor: Hal Cuchman and Stephanie Yelenik}

\item[] Alberto Conti, PhD Student – 2025 - Present. Department of Anthropology, University of Nevada, Reno. Reconstructing human settlement patterns at Lava Beds National Monument, OR \textit{Advisor: Chris Morgan}

\item[] Alexander Audet, PhD Candidate - 2023 - Present, Department of Geography, University of Nevada, Reno. Investigating Global Controls on North American Great Basin Climate by Modeling Paleo-Glacier and Pluvial Lake Systems
 \textit{Advisor: Doug Boyle}

\item[] Oyinbarakemi Ndiomu, PhD Student - 2024 - Present, Department of Geography, University of Nevada, Reno. Differential Heat Island Effects on Human Demographic Groups in the Las Vegas Valley \textit{Advisor: Tom Albright}

\item[] Zinat Ara, PhD Student - 2024 - Present, Department of Geography, University of Nevada, Reno. Remote Sensing of Heat Island Effects in the Las Vegas Valley \textit{Advisor: Tom Albright}

\item[] Griffin Shelor, MS Student – 2024 - Present. Department of Geography, University of Nevada, Reno. Evaluating the Representativeness of NASA Snow Sampling Sites in the Western U.S. and Alaska. \textit{Advisor: Anne Nolin}

\item[] Elizabeth Everest, MS 2023 - Present. Program in Ecology Evolution and Conservation Biology, University of Nevada, Reno. Quantifying Fishing Effort in South East Asia's Largest Lake
 \textit{Advisor: Zeb Hogan}

\item[] Shelby Saper, PhD Student – 2023 - Present. Department of Anthropology, University of Nevada, Reno. Reconstructing human populations on Channel Islands, CA \textit{Advisor: Chris Morgan}

\item[] Sean Begg, PhD Student – 2023 - Present. Department of Anthropology, University of Nevada, Reno. Reconstructing human populations in the Great Basin of Northern Nevada \textit{Advisor: Chris Morgan}

\item[] Johanne Albrigtsen, PhD Candidate – 2022 - Present. Graduate Program of Hydrologic Sciences, University of Nevada.  Do feedback cycles between snow accumulation, soil edaphics, and vegetation growing conditions drive subarctic ecosystem conversions? \textit{Advisor: Scott Allen}

\item[] Colton Irons, MS Student – 2023 - Present. Department of Natural Resource and Environmental Science, University of Nevada. Monitoring habitat use by rattlesnakes and their rodent prey in a fire-managed ecosystem in eastern Nevada  (FVS)\textit{Advisor: Kevin Shoemaker}

\item[] Ellen Brecke (Wilson), MS Student – 2021 - Present. Department of Biology, University of Nevada. Modeling spotted frog habitat in the Great Basin. \textit{Advisor: Mary Peacock}
 
 \item[] Rocky Brockway, PhD Candidate – 2020 – Present. Department of Anthropology, University of Nevada, Reno. Spatial Analysis of Native American settlement sites in the Sierra Nevada Mountains. \textit{Advisor: Chris Morgan}

\item[] Theo Hartsook, PhD Candidate – 2022 - Present. Department of Natural Resource and Environmental Science, University of Nevada. Modeling Snow depth using machine learning techniques. \textit{Advisor: Jonathan Greenberg}

\item[] Taylor Brown, PhD Candidate – 2023 - Present. Department of Natural Resource and Environmental Science, University of Nevada. Modeling synthetic tree lists for fire simulation scenarios in the Forest Vegetation Simulator (FVS)\textit{Advisor: Jonathan Greenberg}

 \item[] James Simmons, PhD Candidate – 2019 – Present. Program in Ecology Evolution and Conservation Biology, University of Nevada, Reno. Spatial analysis of productivity in inland forest lakes. \textit{Advisor: Sudeep Chandra}
 
  \item[] Erica Bradley, PhD Candidate – 2019 – Present. Department of Anthropology, University of Nevada, Reno. Spatial Analysis of Native American tool deposition sites. \textit{Advisor: Geoff Smith}

  \item[] Jon DeBoer, PhD Candidate – 2018 – Present. Department of Geography, University of Nevada, Reno. Modeling habitat and genetics for Girdled Lizards in Namibia. \textit{Advisor: Jill Heaton}
   
  
  \textit{Graduated}
\item[] Melissa Schwann, MS 2025, Department of Geography, University of Nevada, Reno. Creating field-based earth science curriculum \textit{Advisor: Adam Csank}

\item[] Julia Brockman, PhD 2025. Program in Ecology Evolution and Conservation Biology, University of Nevada, Reno. Ravens, movement ecology, and conservation. \textit{Advisor: Perry Williams}

\item[] Ericka Campos, PhD 2025. Department of Civil & Environmental Engineering, University of Nevada, Reno. Advanced Mobility Analytics Applications Using High-resolution Vehicle Trajectory Data \textit{Advisor: Zong Tian}
   
   \item[] Ally Coconis, PhD 2025. Program in Ecology Evolution and Conservation Biology, University of Nevada, Reno. Habitat and genetics of Neotoma species in the Great Basin. \textit{Advisor: Marjorie Matocq}

  \item[] Robert Chastain, PhD 2025. Department of Statistics and Data Science, University of Nevada, Reno. Examining truncated distributions for predictive modeling \textit{Advisor: Anna Panorska}
  
   \item[] Ranae Sullivan, PhD 2024. Program in Ecology Evolution and Conservation Biology, University of Nevada, Reno. Effectiveness of herbicides in controlling invasive grasses in the Mojave desert. \textit{Advisor: Beth Newingham}

    \item[] Elizabeth Mast, MS 2024. Program in Ecology Evolution and Conservation Biology, University of Nevada, Reno. Climate Driven Variation in Ice Out Impacts Zoobenthos Biodiversity and Production in a Subalpine Lake \textit{Advisor: Sudeep Chandra}
   
   \item[] Jack Tarricone, PhD 2023. Department of Geography, University of Nevada. Modeling Snow depth using machine learning techniques. \textit{Advisor: Anne Nolin}
   
  \item[] Lauren Phillips, PhD 2023. Department of Geography, University of Nevada, Reno. Modeling habitat for Southwestern Willow Flycatcher. \textit{Advisor: Tom Albright}
  
\item[] Dominic Tullo, MS 2023 Department of Anthropology, University of Nevada, Reno. Temporal analysis of native American settlements near Grass Valley, CA. \textit{Advisor: Chris Morgan}

\item[] Mara Huang, MS 2023. Department of Environmental Sciences, University of Nevada. Effects of grazing on vegetation. \textit{Advisor: Robert Washington-Allen}

\item[] Ezra Jones, MS 2023. Department of Geography, University of Nevada. Reclaiming disturbed lands for use in renewable energy production. \textit{Advisor: Scott Bassett}

\item[] James Golden, MS 2023. Natural Resource and Environmental Science, University of Nevada. The effects of upland riparian meadow health and its associated effects on the population dynamics of adult and juvenile sage grouse. \textit{Advisor: Perry Williams}
  
  \item[] Derek Friend, PhD 2022. Department of Geography, University of Nevada, Reno. Using simulations to predict the genetic connectivity of the Mojave desert tortoise. \textit{Advisor: Scott Bassett}
  
  \item[] Margarete Walden, PhD 2022. Program in Ecology Evolution and Conservation Biology, University of Nevada, Reno. Body Condition, Hatchling Sex Identification, and Somatic Growth of the Mojave Desert Tortoise, Gopherus agassizii. \textit{Advisor: Kevin Shoemaker}
   
   \item[] Wade Plafcan, MS 2022. Department of Geography, University of Nevada, Reno. Recent and Future Climate-induced Range Shifts of Four High-elevation Rare Forbs in Great Basin National Park, USA. \textit{Advisor: Tom Albright}
   
   \item[] Meghan Keating, MS – 2021. Natural Resources and Environmental Sciences, University of Nevada, Reno. Analysis of Mule Deer Movements in the Great Basin. \textit{Advisor: Perry Williams}
   \item[] Sam Cartwright, MS 2021. Dept. of Geologic Sciences and Engineering, University of Nevada, Reno.  Compositional Investigation of Stratigraphic and Morphologic Units in the South Polar Ice Deposits of Mars. \textit{Advisor: Wendy Calvin}
   \item[] Alice Ready, MS – 2021. Natural Resources and Environmental Sciences, University of Nevada, Reno. Using UAS to detect the invasive Medusa Head in Northern Nevada. \textit{Advisor: Peter Weisberg}
   \item[] Scott Wright, MS 2021. Department of Geography, University of Nevada, Reno. What are the trade-offs and land consumption levels of rapidly converting to large- scale utility renewables. \textit{Advisor: Scott Bassett}
   \item[] Alex Greenwald, MS 2020. Department of Geography, University of Nevada, Reno. Life Cycle of Snow in the Sierra Nevada Mountains of California. \textit{Advisor: Anne Nolin}
   \item[] Dillon Monroe, MS 2020. Department of Biology, California State University, Northridge, CA. Challenges to Defining the Niche of an Invasive Human Commensal Gecko. \textit{Advisor: Robert Espinoza}
   \item[] Rui Yuie, PhD 2020. Center for Advanced Transportation Education and Research, University of Nevada, Reno.  Effects of light timing and spacing on traffic flow. \textit{Advisor: Tian Zong}
   \item[] Amy Robards, MS 2019. Department of Mathematic and Statistics, University of Nevada, Reno. Absolute Frequency Data versus Sample Data: Increasing Calculation Efficiency in Practical Statistics. \textit{Advisor: Paul Hurtado}
   \item[] Kaitlyn Weber, MS 2018. Department of Geography, University of Nevada, Reno. Modern Role of Internal Decadal Variability within the Arctic Climate System. \textit{Advisor: Stephanie McAfee}
   \item[]Christina Aiello, PhD 2018. Department of Ecology, Penn State University. Invasion and Infection: Translocation and Transmission: An experimental study with \textit {Mycoplasma} in Desert Tortoises. \textit{Advisor: Peter Hudson}
   \item[]Pratha Sah, PhD 2017. Biology Department, Georgetown University. Dissertation – Predicting the impact of stressors on dynamics of future infection spread in desert tortoise population using network analysis. \textit{Advisor: Shweta Bansal}
   \item[]Kristina Drake, PhD 2017. Joint Doctoral Program in Ecology, San Diego State University and UC Davis. Using gene expression to evaluate the influence of habitat quality on animal physiology. \textit{Advisors: Rebecca Lewison and Keith Miles}
  \item[] Anna Patterson, PhD 2016. Department of Geography, University of Nevada Reno. Native Americans and forest composition: a paleoecologic investigation into the impacts of fire and landscape modification from two wet meadows in the southern Sierra Nevada, California. \textit{ Advisor: Scott Mensing}
   \item[]Sarah Snyder. PhD 2014. Program in Ecology, Evolution and Conservation Biology. Effects of fire on desert tortoise thermal ecology. University of Nevada, Reno. \textit{ Advisor: C. Richard Tracy}
   \item[]Pete Noles, MS 2010. Department of Geography. University of Nevada, Reno. Reconciling Western toad phylogeography with Great Basin pre-history. \textit{Advisor: Jill Heaton}\\
   
\end{itemize}
\end{rSubsection}
\begin{rSubsection}{Undergraduates Supervised}{}{}{}
\begin{itemize}

\item[] Yareli Navarro Chavez, 2024. Veterinary Science. University of Nevada, Reno. EPSCOR grant - Office of Undergraduate Research. Research Project: Changing elephant behavior to reduce the human/elephant conflict in Sri Lanka.

\item[] Yasmin Perry, 2024. Veterinary Science. University of Nevada, Reno. EPSCOR grant - Office of Undergraduate Research. Research Project: Changing elephant behavior to reduce the human/elephant conflict in Sri Lanka.

  \item[] Dahlia Villalobos, 2023 - 2024. Veterinary Science. University of Nevada, Reno. PREP award - Office of Undergraduate Research. Research Project: Species habitat modeling for species of concern in Clark County NV.
  
  \item[] Jorge Esparza, 2023 - 2024. Wildlife Ecology & Conservation. University of Nevada, Reno. EPSCOR grant - Office of Undergraduate Research. Research Project: Changing elephant behavior to reduce the human/elephant conflict in Sri Lanka.
  
  \item[] Joshua Schein, BS - 2020. Department of Geography. University of Nevada, Reno. Research Project: Truckee Meadows Nature Study Area: Vegetation Cover.
  
  \item[] Kebrina Rosino, BS - 2018. Department of Geography. University of Nevada, Reno. Research Project: Applications of unmanned aerial systems of UAS for monitoring the topography of arid basins and rivers in the Riparian region. \end{itemize}

\end{rSubsection}
\end{rSection}

 %------------------------------------------------
% Teaching Experience
%------------------------------------------------
\begin{rSection}{Teaching Experience}
Instructor of Record, University of Nevada, Reno \hfill 2017 - Present
\begin{center}
\begin{tabular}{p{0.4\linewidth}p{0.4\linewidth}}
   $\bullet$  \textit{Advanced GIS, Lecture and Laboratory -- GEOG 407/607} &
   $\bullet$  \textit{Spatial Analysis, Lecture and Laboratory -- GEOG 416/616} \\
   
   $\bullet$  \textit{GIS Design Studio (UAS Operations), Lecture and Laboratory -- GEOG 409/609} &
   $\bullet$  \textit{Application of UAS technologies in Geography, Lecture and Laboratory -- GEOG 413/613} \\
   
   $\bullet$  \textit{Species Distribution Modeling, Lecture and Laboratory -- GEOG 701M} &
    $\bullet$  \textit{Biogeography, Graduate Seminar -- GEOG 701S}\\
    
    $\bullet$  \textit{Environmental Planning and Policy -- GEOG 466/666} &
 $\bullet$ \textit{Geography Colloquium, Speaker Series -- GEOG 490/690} \\
 
 $\bullet$ \textit{GIS Design Studio (Thematic real world problems w GIS), Lecture and Laboratory -- GEOG 409/609} \\


\end{tabular}
\end{center}
\\

Guest Lecturer, University of Nevada, Reno \hfill 2016 - 2022
\begin{center}
\begin{tabular}{p{0.4\linewidth}p{0.4\linewidth}}
 $\bullet$ \textit{History and Nature of Geography -- GEOG 700} &
 $\bullet$ \textit{Principles of Ecology, Evolution and Conservation Biology -- EECB 703} \cr \cr
   $\bullet$ \textit{Research Methods -- GEOG 325/313} &
   $\bullet$ \textit{Intro to Geotechnology -- GEOG 210} \cr \cr 
   $\bullet$ \textit{Intro to GIS -- GEOG 405/605} &
   $\bullet$ \textit{Remote Sensing -- GEOG 411/611}
   \\
   $\bullet$ \textit{Elements of Research Computing -- GRAD 778}
   \\
\end{tabular}
\end{center}

Guest Lecturer, University of Nevada, Las Vegas \hfill 2009, 2011
\begin{center}
\begin{tabular}{p{0.4\linewidth}p{0.4\linewidth}}
		$\bullet$ 	\textit{Principles of Ecology} &
    $\bullet$ 	\textit{Introduction to Human Ecology} \\
\end{tabular}
\end{center}

Guest Lecturer, University of Nevada, Reno \hfill 2002
\begin{center}
\begin{tabular}{p{0.4\linewidth}p{0.4\linewidth}}
   $\bullet$  \textit{Desert and Montane Ecosystems} & \\
\end{tabular}
\end{center}

Co-Instructor, University of Nevada, Reno \hfill 1998 and 2000
\begin{center}
\begin{tabular}{p{0.4\linewidth}p{0.4\linewidth}}
		$\bullet$ 	\textit{Herpetology: Lecture and Laboratory} &
\end{tabular}
\end{center}

Graduate Teaching Assistant, Colorado State University \hfill 1994 to 1995
\begin{center}
\begin{tabular}{p{0.4\linewidth}p{0.4\linewidth}}
	   $\bullet$ 	\textit{Animal Ecology Laboratory} & 
     $\bullet$ 	\textit{Vertebrate Biology Laboratory} \\
\end{tabular}
\end{center}

Undergraduate Teaching Assistant, Colorado State University \hfill 1993 to 1994
\begin{center}
\begin{tabular}{p{0.4\linewidth}p{0.4\linewidth}}
	   $\bullet$ \textit{Vertebrate Biology Laboratory} & 
     $\bullet$ \textit{Introductory Zoology Laboratory} \\
\end{tabular}
\end{center}

\end{rSection}
%------------------------------------------------
% Professional Service
%------------------------------------------------

\begin{rSection}{Professional Service}
\begin{rSubsection}{Professional Committees}{}{}{}
\begin{itemize}

\item \textit{Host Committee - IALE - North America 2021 Annual Meeting} \hfill 2020 - 2021 \\
Planning and hosting duties for 2021 Annual Meeting of the International Association of Landscape Ecologists
\item \textit{Comprehensive Examination Committee - Program in Ecology, Evolution and Conservation Biology, UNR} \hfill 2018 - 2022 \\
Comprehensive Exam Preparation and Evaluation.
\item \textit{Personnel Committee - Department of Geography, UNR} \hfill 2020 \\
Search committee member for Teaching Assistant Professor in GIS.
\item \textit{Search Committee - Department of Geography, UNR} \hfill 2019 - 2020 \\
Search committee member for Teaching Assistant Professor in GIS.
\item \textit{Board Member - University of Nevada Cyberinfrastructure Committee} \hfill 2018 - Present \\
The Cyberinfrastructure Committee (CiC) is a representative committee of faculty from all UNR colleges and units. The CiC's mission is to provide a critical perspective and vision for cyberinfrastructure needs across campus, how the University can use cyberinfrastructure to transform the scale and impact of its research, and how a campus cyberinfrastructure program can be developed and most productively used.

\item \textit{Peer Reviewer - US Fish and Wildlife Service - Joshua Tree Species Status Assessment} \hfill 2018 \\
Peer reviewer for USFWS Joshua tress species status assessment to address the listing petition for the species. 
\item \textit{Peer Reviewer - National Park Service - Gulf Coast Network Texas Tortoise Monitoring in Palo Alto Battlefield National Historical Park
} \hfill 2018 \\
Peer reviewer for the monitoring program for the Texas tortoise in National Park properties. Made recommendations on monitoring and data needs for the species.

\item \textit{Science Review Panel member for the Desert Renewable Energy Conservation Plan (DRECP) (BLM)} \hfill 2012 - 2014\\
Peer review panel member for the draft DRECP

\item \textit{Advisory scientist - Nevada Wildlife Action Plan} \hfill 2010 \\
Nevada Department of Wildlife and attended by USGS, NDOW, UNLV, UNR, TNC, USFWS, BLM, DoD, EPA and National Audubon Society. This workshop was designed to develop the action plan for use in desert habitats of southern Nevada.October 10-11, 2010. Provided input into planning and monitoring methods 
\item \textit{Climate Change Adaptation in the Arid Southwest} \hfill 2010 \\ 
A Workshop for Land and Resource Management- Sky Island Alliance – multiple agency workshop to identify science needs and management options in the face of changing climates in the desert southwest. Breakout panel member.
\item \textit{Science Advisory Council} \hfill 2009-2012 \\
 US Geological Survey Western Ecological Research Center
\item \textit{SSAR Graduate Student Workshop} \hfill 2009 \\ 
How to Get a Job After Graduation: Advice from Experts. Presenter at graduate student workshop. Joint Meeting of the American Society of Ichthyologists and Herpetologists, Society for the Study of Amphibians and Reptiles, and the Herpetologists' League, Portland, Oregon USA
\item \textit{Herpetologist League Graduate Research Award} \hfill 2009 \\ 
Graduate Student Award Committee Member, Judged student papers, Joint Meeting of the American Society of Ichthyologists and Herpetologists, Society for the Study of Amphibians and Reptiles, and the Herpetologists' League, Portland, Oregon USA
\item \textit{Organ Pipe National Monument} \hfill 2006 \\
Member of peer review committee for Organ Pipe National Monument Long Term Environmental Monitoring Program, Tucson AZ. Conducted reviews of the 10 year multi-species monitoring program conducted by biologists at Organ Pipe National Monument.
\item \textit{US Fish and Wildlife Service} \hfill 2005 to 2006 \\ Member of Fish and Wildlife Science Team for conservation planning within desert tortoise critical habitat in Coyote Springs, Nevada. Contributed toward a white paper describing conservation and research needs in light of proposed construction of urban development in desert tortoise critical habitat.
\item Fish and Wildlife Desert Tortoise Monitoring Committee. \hfill 2005-2007\\
Assisted in research design, monitoring strategies and analysis of transect sampling data for range-wide monitoring of the Mojave desert tortoise.
\item \textit{US Fish and Wildlife Service} \hfill 2003 to 2004 \\
Member of the U.S. Fish and Wildlife Service Desert Tortoise Recovery Plan Assessment Committee. Conducted literature reviews, data compilation and analyses, co-authored report, and gave presentations to Desert Tortoise Management Oversight Group, USFWS, and Assistant Secretary of the Interior.
\item \textit{Powdermill} \hfill 1999 \\ 
Staff member at Powdermill freshwater turtle conference, Laughlin, NV. – Served on conference coordination staff

\end{itemize}

\end{rSubsection}

\begin{rSubsection}{Professional Societies and Service}{}{}{}

\item \textit{Chelonian Section Co-Editor} - \textit{Herpetological Conservation and Biology} \hfill 2017 - Present

\textbf{\textit{Referee for Scientific Journals:}}

Acta Oecologica, Animal Biotelemetry, Applied Herpetology, Biodiversity and Conservation, Biological Conservation, Biological Reviews, Canadian Journal of Zoology, Chelonian Conservation and Biology,
Condor, Conservation Biology, Copeia, European Journal of Wildlife Research, Eco-Health, Herpetologica, Herpetological Conservation and Biology, Herpetological Journal, Journal of Herpetology, Herpetological Review, Journal of Theoretical Biology, Journal of Thermal Biology, Southwestern Naturalist, Journal of Wildlife Biology, The Journal of Wildlife Management, Royal Society Open Science, Wildlife Research, Wildlife Society Bulletin 

\end{rSubsection}
%------------------------------------------------
% National Service
%------------------------------------------------
\begin{rSection}{National Service}

\textbf{Infantry Paratrooper}, Sergeant, US Army, 82nd Airborne Division \hfill 1987 to 1991 \\ 
Combat deployments in Panama, and Iraq. Honorably Discharged.

\textit{Awards Earned} \\
\begin{tabular}{p{0.5\linewidth}p{0.5\linewidth}}
\tabularnewline
\textbullet Army Achievement Medal & \textbullet Parachutist Badge with Bronze Service Star\\
\textbullet Army Good Conduct Medal & \textbullet Honduran Parachutist Badge\\
\textbullet National Defense Service Medal & \textbullet Armed Forces Expeditionary Medal (Panama)\\
\textbullet Combat Infantryman Badge (with service stars) & \textbullet Southwest Asia Service Medal with two Bronze Service Stars\\
\textbullet Expert Infantryman Badge & \textbullet Kuwait Liberation Medal\\
\end{tabular}

\textbf{Military Academy Nominations Committee} for Senator Jacky Rosen (NV). \hfill 2020 - 2021 \\
Reviewed applications and interviewed applicants from northern Nevada to the National Service Academies (West Point, Air Force Academy, Annapolis Naval Academy)


\end{rSection}
%------------------------------------------------
% Community Service
%------------------------------------------------

\end{rSection}
\begin{rSection}{Community Service}
\textit{Air Races Tram Driver} Experimental Aircraft Association - Reno National Air Races, Reno NV \hfill 2022-2024 \\
\textit{Nevada Promise Mentor} Truckee Meadows Community College, Reno NV \hfill 2019 \\
\textit{Veterans Day Presenter} Cottonwood Elementary School, Fernley, NV \hfill 2019 \\
\textit{Martial Arts Instructor} \\
\textit{Head Instructor}, Bushidokan Jujitsu, Sparks, NV \hfill 2016 - 2020\\
\textit{Head Instructor (Sensei)}, Desert Breeze Jujitsu, Las Vegas, NV \hfill 2006 to 2015\\
\textit{Instructor}, Boulder City Jujitsu, Boulder City, NV \hfill 2004 to 2006

\textit{“60 Million Years”} \hfill 2014\\
Opening talk for new film on Desert Tortoises. Las Vegas Tortoise Group, and Boulder City Historical Society.

\textit{Career Day Presenter} \hfill 2006 \\
Estes McDoniel Elementary School, Henderson NV

\textit{Amphibians and Reptiles of Nevada} \hfill 2003 \\
Brown Elementary School, Reno NV 

\textit{Conservation of Amphibians and Reptiles} \hfill 2003 \\
University of Nevada Biology Outreach Program, South Lake Tahoe Middle School

\textit{Amphibians and Reptiles of Nevada} \hfill 2002 and 2003 \\
Nevada Day Educational Program, for Cottonwood Elementary School, Fernley NV

\textit{Amphibians and Reptiles of Nevada} \hfill 1999 \\
Our Lady of the Snows Elementary School, Reno NV\\

\end{rSection}
%------------------------------------------------
% Awards
%------------------------------------------------
\begin{rSection}{Honors and Awards}{}
\begin{itemize}
\item Stephen Jenkins Mentorship Award, Ecology, Evolution and Conservation Biology, University of Nevada Reno\hfill 2025
\item Robert C. Stebbins Research Award,  Desert Tortoise Council - \hfill 2025
\item University of Nevada, Reno Honors College - Excellence in Student Mentorship Award \hfill 2024
\item Inman et al. 2023 - Best paper of 2023 for The Wildlife Society \hfill 2024
\item Dilts et al. 2016 - Honorable mention for best paper at the US Chapter of the International Association of Landscape Ecology \hfill 2017
\item Heaton et al. 2008 - recommended reading for Wildlife Professionals by the Editorial Advisory Board for Science in Short Spring 2009 The Wildlife Professional. \hfill 2009
\item Best Student Presentation (\$200) \hfill 2004
Annual Symposium of the Desert Tortoise Council, Las Vegas, NV. Can modeling of tortoise activity be used to improve species monitoring?
\item Beta Sigma Phi International Scholarship (\$1000) \hfill
1994 
Inducted into Phi Beta Kappa, Phi Kappa Phi, and Pinnacle national honor societies, Colorado State University (C.S.U.)
\item President's Scholarship Colorado State University (\$1000) \hfill 1993 
\item College of Natural Sciences Alumni Merit Scholarship, C.S.U. (\$1000) \hfill 1993 
\item Dean's List, Colorado State University College of Natural Sciences \hfill 1992 to 1995
Awarded every semester
\end{itemize}

\end{rSection}
%------------------------------------------------
% Publications
%------------------------------------------------
%\rhead{Nussear - Curriculum Vitae \hfill Publications Cont.}

%\clearpage
%\pagebreak
\begin{rSection}{Peer-reviewed Publications}
\begin{refsection}[bibdb]
\nocite{*}
\defbibnote{janote}{(* Indicates Student Advised)}
\printbibliography[type=article,heading=subbibliography,title={Journal Articles}, prenote=janote,keyword={journal}]
%\rhead{Nussear - Curriculum Vitae \hfill Publications Cont.}
\printbibliography[heading={subbibliography},title={Book Chapters},type=incollection,keyword={book}]
\end{refsection}

\begin{refsection}[bibdb]
\nocite{*}
\printbibliography[heading={subbibliography},title={Technical Reports - \textit{Peer Reviewed}},keyword={report}]
\end{refsection}

\end{rSection}
\end{document}
